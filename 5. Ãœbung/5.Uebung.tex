\documentclass[12pt]{article}
\usepackage{amsfonts}
\usepackage{fancyhdr}
\usepackage[a4paper, top=2.5cm, bottom=2.5cm, left=2.2cm, right=2.2cm]{geometry}
\usepackage{times}
\usepackage{amsmath}
\usepackage{changepage}
\usepackage{amssymb}
\usepackage{graphicx}%
\setcounter{MaxMatrixCols}{30}
\newtheorem{theorem}{Theorem}
\newtheorem{acknowledgement}[theorem]{Acknowledgement}
\newtheorem{algorithm}[theorem]{Algorithm}
\newtheorem{axiom}{Axiom}
\newtheorem{case}[theorem]{Case}
\newtheorem{claim}[theorem]{Claim}
\newtheorem{conclusion}[theorem]{Conclusion}
\newtheorem{condition}[theorem]{Condition}
\newtheorem{conjecture}[theorem]{Conjecture}
\newtheorem{corollary}[theorem]{Corollary}
\newtheorem{criterion}[theorem]{Criterion}
\newtheorem{definition}[theorem]{Definition}
\newtheorem{example}[theorem]{Example}
\newtheorem{exercise}[theorem]{Exercise}
\newtheorem{lemma}[theorem]{Lemma}
\newtheorem{notation}[theorem]{Notation}
\newtheorem{problem}[theorem]{Problem}
\newtheorem{proposition}[theorem]{Proposition}
\newtheorem{remark}[theorem]{Remark}
\newtheorem{solution}[theorem]{Solution}
\newtheorem{summary}[theorem]{Summary}
\usepackage{enumitem}
\usepackage[utf8]{inputenc}
\newenvironment{proof}[1][Proof]{\textbf{#1.} }{\ \rule{0.5em}{0.5em}}
\usepackage{tikz}
\usetikzlibrary{fit,arrows,calc,positioning}
\usepackage{graphicx}
\usepackage{wrapfig}
\usepackage{float}
\usepackage{datetime}
\newdateformat{specialdate}{\twodigit{\THEDAY}.\twodigit{\THEMONTH}.\THEYEAR}
\usepackage{amssymb}
\usepackage{ifsym}
\usepackage{mathtools}
\usepackage{listings}
\usepackage{lipsum}  

\newcommand{\blue}[1]{\color{blue}#1\color{black}}
\newcommand{\red}[1]{\color{red}#1\color{black}}
\newcommand{\green}[1]{\color{green}#1\color{black}}
\newcommand{\N}{\mathbb{N}}
\newcommand{\Q}{\mathbb{Q}}
\newcommand{\R}{\mathbb{R}}
\newcommand{\C}{\mathbb{C}}
\newcommand{\Z}{\mathbb{Z}}
\renewcommand\labelenumii{\theenumi.\arabic{enumii}.}

\begin{document}
	
	\title{5. Exercise}
	\author{Timo Bergerbusch 344408 \\ Thomas Näveke 311392 \\ Shu Xing 381176}
	\date{\specialdate\today}
	\maketitle
	
	\section*{Exercise 5.1}
	Important numbers:\\
	\begin{itemize}
		\item[\textbf{B}:] $\frac{20\,000\,000 \text{ Records}}{40 \frac{\text{Records}}{\text{Page}}}= 500\,000 \text{ Pages}$
		\item[\textbf{R}:] $40 \frac{\text{Records}}{\text{Page}}$
	\end{itemize}
	\subsection*{1.a) $\sigma_{ID=500}$}
		\underline{Answer:} Use path A3. \\
		\underline{Explanation:} The unclustered hash index needs only $2D$ I/Os. All other produce higher costs. The clustered $B^+$ tree index would need a $G$ like the following
		\begin{align*}
			D\cdot(log_G 0.15B) &\le 2D \\        
			log_G 75000 &\le 1 \\
			4.261159991\cdot10^{-411955} &\le G < 1\\
		\end{align*}\\
		Such a $G$ is not possible as a tree-branch-factor. So the unlustered hash index is most efficient.
	\subsection*{1.b) $\sigma_{ID\neq 500}$}
		\underline{Answer:} Use path A2\\
		\underline{Explanation:} \\
			\begin{tabular} {r | l |l}
				type & formula & cost \\ \hline
				sorted & $B\cdot D$ & $500\,000D$ \\
				$B^+$ tree & $0.15BD+1.5BD$ & $825\,000D$ \\
				hash index & $BD(R+0.125)$ & $20\,062\,500D$
			\end{tabular}
	\subsection*{1.c) $\sigma_{ID>500 \land ID<1500}$}
		\underline{Answer:} \\
		\underline{Explanation:} \\
			For the sorted file and the clustered has index the cost would be $B\cdot D = 500\,000D$. For a clustered tree index to be cheaper we have to consider the two variables $G$ as the branching factor of the $B^+$ tree. The number of matching pages is given by $1500-500=1000$. The cost can be computed as $D\cdot (\log_G 0.15B+\# matchingPages) \Rightarrow D\cdot (\log_G 75\,000 + 1000)$. Now for the tree index to be cheaper this has to be $< 500\,000D$.
			\begin{align*}
				D\cdot (\log_G 75\,000 + 1000) &< 500\,000D \\
				\log_G 75\,000 + 1000 &< 500\,000 \\
				\log_G 75\,000 &< 499\,000 \\
				G &< \sqrt[499\,000]{75\,000} \approx 1.00002
			\end{align*}\\
			%TODO: seems to be false
	
	\subsection*{2.a)}
		%TODO
	\subsection*{2.b) i.}
		%TODO
	\subsection*{2.b) ii.}
		%TODO
	\subsection*{2.b) iii.}
		%TODO
	\section*{Exercise 5.2}
	\subsection*{1.}
		Split the computing into two parts:
		\begin{itemize}
			\item Finding the first suitable records\\
				\begin{enumerate}
					\item through the stated average size and load we retrieve $0.15\cdot B$ data entries
					\item the height of such a tree then is given by $log_G 0.15\cdot B$
					\item Since every one has to be read we multiply it by $D$
					\item[$\Rightarrow$] $D\cdot(log_G 0.15\cdot B)$
				\end{enumerate}				
			\item find first following non-suitable record 
				We then continue reading the index structure until we encounter a record that is not suitable. Through the clustering we know that there are no suitable records following. So we read the number of suitable records = $\# matchingPages$ many records\\
				$\Rightarrow$ $D\cdot \# matchingPages$
			\item[$\Rightarrow$] $D\cdot(log_G 0.15\cdot B) + D\cdot \# matchingPages = D\cdot(log_G 0.15\cdot B + \# matchingPages)$
		\end{itemize}
	\subsection*{2.}
		\begin{itemize}
			\item read the page of the hash-group, which contains all entries that satisfy the equality $\Rightarrow D$
			\item read the data entry satisfying the equality $\Rightarrow D$
			\item[$\Rightarrow$] $D+D=2D$
		\end{itemize}
	\subsection*{3.}
		\begin{itemize}
			\item the costs of retrieving the file containing the data computes as in Exercise 5.2.1 to $D\cdot (\log_G 0.15\cdot B)$
			\item the costs of finding the data entry is constant $D$
			\item rewriting the index page and datafile takes $2D$
			\item[$\Rightarrow$] $D\cdot (\log_G 0.15\cdot B) + D + 2D = D\cdot (3+\log_G 0.15\cdot B)$
		\end{itemize}
	
	\section*{Exercise 5.3}
		\begin{itemize}
			\item Regarding \texttt{Q1}: \\
				Equality Selection based on its primary key sno: \\
				\begin{tabular}{r | l | l}
					type & formula & cost \\ \hline
					heap & $\frac{1}{2}BD$ & $500D$ \\
					Sorted & $D\cdot \log_2 B$ & $9.97D$\\
					Clustered Tree Index & $D\cdot (1+ \log_G 0.15B)$ & $2.09D$ \\
					Unclustered Tree Index & $D\cdot (1+ \log_G 0.15B)$ & $2.09D$\\
					Unclustered Hash Index & $2D$ & $2D$ \\
				\end{tabular}
				$\Rightarrow$ for \texttt{Q1} a unclustered hash index on sno would be the best option. A(n) (un)clustered tree index would only be very slightly less efficient.
			\item Regarding \texttt{Q2}: \\
				Range Selection based on the salary:\\
				Obviously a tree index is the most efficient for such queries. Also having a clustered tree index would be more efficient than an unclustered tree index, since otherwise \underline{each} suitable records would cost 1 page I/O. A clustered tree index can read the records continuously which results in less page I/Os.\\
				So a clustered tree index would be the best choice. 
			\item Regarding \texttt{Q3}: \\
				maybe indexing sno? %TODO
		\end{itemize}
\end{document}


















