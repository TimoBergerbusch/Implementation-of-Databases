\documentclass[12pt]{article}
\usepackage{amsfonts}
\usepackage{fancyhdr}
\usepackage[a4paper, top=2.5cm, bottom=2.5cm, left=2.2cm, right=2.2cm]{geometry}
\usepackage{times}
\usepackage{amsmath}
\usepackage{changepage}
\usepackage{amssymb}
\usepackage{graphicx}%
\setcounter{MaxMatrixCols}{30}
\newtheorem{theorem}{Theorem}
\newtheorem{acknowledgement}[theorem]{Acknowledgement}
\newtheorem{algorithm}[theorem]{Algorithm}
\newtheorem{axiom}{Axiom}
\newtheorem{case}[theorem]{Case}
\newtheorem{claim}[theorem]{Claim}
\newtheorem{conclusion}[theorem]{Conclusion}
\newtheorem{condition}[theorem]{Condition}
\newtheorem{conjecture}[theorem]{Conjecture}
\newtheorem{corollary}[theorem]{Corollary}
\newtheorem{criterion}[theorem]{Criterion}
\newtheorem{definition}[theorem]{Definition}
\newtheorem{example}[theorem]{Example}
\newtheorem{exercise}[theorem]{Exercise}
\newtheorem{lemma}[theorem]{Lemma}
\newtheorem{notation}[theorem]{Notation}
\newtheorem{problem}[theorem]{Problem}
\newtheorem{proposition}[theorem]{Proposition}
\newtheorem{remark}[theorem]{Remark}
\newtheorem{solution}[theorem]{Solution}
\newtheorem{summary}[theorem]{Summary}
\usepackage{enumitem}
\usepackage[utf8]{inputenc}
\newenvironment{proof}[1][Proof]{\textbf{#1.} }{\ \rule{0.5em}{0.5em}}
\usepackage{tikz}
\usetikzlibrary{positioning}
\usepackage{graphicx}
\usepackage{wrapfig}
\usepackage{float}
\usepackage{datetime}
\newdateformat{specialdate}{\twodigit{\THEDAY}.\twodigit{\THEMONTH}.\THEYEAR}
\usepackage{amssymb}
\usepackage{ifsym}
\usepackage{mathtools}
\usepackage{listings}
\usepackage{lipsum}  

\newcommand{\blue}[1]{\color{blue}#1\color{black}}
\newcommand{\red}[1]{\color{red}#1\color{black}}
\newcommand{\green}[1]{\color{green}#1\color{black}}
\newcommand{\N}{\mathbb{N}}
\newcommand{\Q}{\mathbb{Q}}
\newcommand{\R}{\mathbb{R}}
\newcommand{\C}{\mathbb{C}}
\newcommand{\Z}{\mathbb{Z}}
\renewcommand\labelenumii{\theenumi.\arabic{enumii}.}

\begin{document}
	
	\title{3. Exercise}
	\author{Timo Bergerbusch 344408 \\ Thomas Näveke 311392 \\ Shu Xing 381176}
	\date{\specialdate\today}
	\maketitle
	
	\section*{Exercise 3.1}
	\subsection*{1. a)}
	
		$\blue{\pi_{A,B}(R)} \bowtie \green{\pi_{B,C}}(\red{\pi_{A,C}(\sigma_{B=1}(R))} \bowtie \blue{\pi_{A,B}(R)})$ \\\\
		$\blue{S_1} := \{ [a,b] \mid \exists c \> R(a,b, c)\}$\\
		$\red{S_2} := \{ [a,c] \mid \exists a,c \> R(a,1,c)\}$\\
		$\green{\pi_{B=1}(S_2 \bowtie S_1)} := \{ [b,c] \mid \exists a,c_1,c_2 \> R(a,b,c_1)\land R(a,1,c_2) \}$ \\
		Insgesamt $:= \{ [a,b,c] \mid \exists a_1,c_1,c_2( \> R(a,b,c_1) \land R(a_1,1,c) \land R(a_1,b,c_2)\}$
	\subsection*{1.b)}
		\begin{figure}[H]
%			\centering
			\hspace{1in}
			\begin{tabular}{c|c|c|c}
				$T_1$ & & & \\ \hline
				$a$ & $b$ & $c_1$ & R \\
				$a_1$ & $1$ & $c$ & R \\
				$a_1$ & $b$ & $c_2$ & R \\
			\end{tabular}
		\end{figure}
	\subsection*{2)}
		
		$h_1: T_2 \rightarrow T_1: a\rightarrow a, b \rightarrow b, a_5 \rightarrow a_1, b_5 \rightarrow b_1, c_4 \rightarrow c_1$\\
		$\Rightarrow T_2 \subseteq T_1\qquad(1)$\\
		\\
		$h_2: T_1 \rightarrow T_2: a\rightarrow a, b \rightarrow b, b_1 \rightarrow b, c_1 \rightarrow c_4, a_1 \rightarrow a, b_2 \rightarrow b_5, c_2 \rightarrow c_4, b_3 \rightarrow b_5, c_3 \rightarrow  c_4$\\
		$\Rightarrow T_1 \subseteq T_2\qquad(2)$\\
		\\		
		$(1) \text{\&}(2) \Rightarrow T_1 \equiv T_2$
	
	\newpage
	
	\section*{Exercise 3.2}
	Given 16 buffer pages (B) and :
	\begin{itemize}
		\item Album has a size of 10.000 pages (M), 40 bytes record size ($s_1$) and 100 tuples/page ($p_A$)
		\item Track has a size of 200.000 pages (N), 30 bytes record size ($s_2$) and 80 tuples/page ($p_T$)
	\end{itemize}
	\subsection*{1)}		
		Since the simple nested loop join is a double iteration over both relations the I/O requirements can be calculated as follows:\\
		\begin{figure}[H]
			\centering
			$M+p_A \cdot M \cdot N = 10.000 + 100 \cdot 10.000 \cdot 200.000 = 200.000.010.000$ I/Os
		\end{figure}
	\subsection*{2)}
		Since the block nested loop join uses 1 input and 1 output buffer the number of I/Os can be calculated with the following formula:
		\begin{figure}[H]
			\centering
			$M + \lceil \frac{M}{B-2} \rceil \cdot N = 10.000 + \lceil \frac{10.000}{16-2} \rceil \cdot 200.000 = 143.060.000$
		\end{figure}
	\subsection*{3)}
		Similarities:
		\begin{itemize}
			\item double loop schema
		\end{itemize}
		Differences:
		\begin{itemize}
			\item usage of buffer pages 
			\item number of \textit{outer} elements in the memory
			\item hashing used in the block nested loop
		\end{itemize}
	
		Explanation:\\
		Lets define the outer loop to be over E and the inner loop to be over T.\\
		The \textit{block nested loop join} loads up to $B-2$ pages of E into the memory. These blocks get stored in a hash table. Now every input of a page from T can be used via one of the two reserved pages. After hash probing the input, if there is any match in a block of Es, we can reduce the number of scanned Es per T and therefore reduce I/Os.
		
	\section*{Exercise 3.3}
	\subsection*{1.}
		\begin{itemize}
			\setlength{\itemindent}{1cm}
			\item[\textbf{Emp:}] $4.000\text{ byte}/20\text{ byte}= 200 \Rightarrow$ 200 records fit on one page\\
						$20.000\text{ records}/200\frac{\text{records}}{\text{page}}= 100 \Rightarrow$ need 100 pages
			\item[\textbf{DeptProj:}] $4.000\text{ byte}/40\text{ byte}= 100 \Rightarrow$ 100 records fit on one page\\
							 $5.000\text{ records}/100\frac{\text{records}}{\text{page}}= 50 \Rightarrow$ need 50 pages
			\item[\textbf{Proj:}] $4.000\text{ byte}/2.000\text{ byte}= 2 \Rightarrow$ 2 records fit on one page\\
						 $1.000\text{ records}/2\frac{\text{records}}{\text{page}}= 500 \Rightarrow$ need 500 pages
		\end{itemize}
	\subsection*{2.}
%		With the number of data pages $B=12$, number of matching records $N$ and the time to read or write $D=3$ we compute
%			$$ B\cdot D > D\cdot (\log_G 0.15B + N) $$
%			$$\Rightarrow 12\cdot 3 > 3 \cdot (\log_G 0.15*12 + N) $$ 
%			$$ $$
		Worst case: have to access all pages $\xRightarrow{1.} 100$ pages\\
		Index access costs 3 $\Rightarrow$ for $N > 97$ scan is cheaper
	\subsection*{3.}
		\begin{enumerate}[label=(\alph*)]
			\item Cost of a \textit{block nested loop join}: $M+\lceil \frac{M}{B-2}\rceil \cdot N$\\
				  $\xRightarrow{1.} M=50, N=100, B=12$, so \\
				  $50+\lceil\frac{50}{12-2}\rceil \cdot 10 = 550$ I/Os
			\item Cost of a \textit{index nested loop join}: $M+((M\cdot p_T)\cdot \text{cost of finding matching E tuples})$\\
				  $\xRightarrow{1.} M=50, p_T= 100, \text{cost of finding matching E tuples}=3+1=4$\\
				  $50+((50\cdot 100) \cdot 4) = 20.050$ I/Os
			\item The best option would be the \textit{Sort-Merge Join} with costs of: $M\log M + N \log N + (M+N)$\\
				  Since they are already sorted we can neglect the sorting cost of $M\log M + N \log N$ and have remaining costs of $M+N$\\
				  $\xRightarrow{1.} M=50, N=100$\\
				  $50+100=150$ I/Os
			\item Identical to 3.3.3.c, since the B+ tree is just a special form of sorting. So analogously we receive the costs $M+N$\\
				  $\xRightarrow{1.} M=50, N=100$\\
				  $50+100=150$ I/Os
		\end{enumerate}
		
\end{document}


















